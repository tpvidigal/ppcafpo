\documentclass[conference,compsoc]{IEEEtran}

% Packages used for adding comments
\usepackage{listings}
\usepackage{xcolor}
\usepackage[portuguese]{babel}
\selectlanguage{portuguese}
\usepackage[T1]{fontenc}
\usepackage[utf8]{inputenc}
\usepackage[colorlinks]{hyperref}
\usepackage[colorinlistoftodos]{todonotes}
\usepackage[numbers]{natbib} % para \citetauthor{} e outros
\usepackage{booktabs}
\usepackage{multirow}
\usepackage{graphicx}
\usepackage{amsmath}

% Python code style
\definecolor{codegreen}{rgb}{0,0.6,0}
\definecolor{codegray}{rgb}{0.5,0.5,0.5}
\definecolor{codepurple}{rgb}{0.58,0,0.82}
\definecolor{backcolour}{rgb}{0.95,0.95,0.92}
\lstdefinestyle{pythoncode}{
    backgroundcolor=\color{backcolour},   
    commentstyle=\color{codegreen},
    keywordstyle=\color{magenta},
    numberstyle=\tiny\color{codegray},
    stringstyle=\color{codepurple},
    basicstyle=\ttfamily\footnotesize,
    breakatwhitespace=false,         
    breaklines=true,                 
    captionpos=b,                    
    keepspaces=true,                 
    numbers=left,                    
    numbersep=5pt,                  
    showspaces=false,                
    showstringspaces=false,
    showtabs=false,                  
    tabsize=2
}
\lstset{style=pythoncode}

% correct bad hyphenation here
\hyphenation{op-tical net-works semi-conduc-tor}

\begin{document}

\title{Exemplos de resolução de problemas lineares utilizando o \textit{solver} do \textit{SciPy}}

\author{\IEEEauthorblockN{Sandro Leite Furtado}
\IEEEauthorblockA{Mestrado em Computação Aplicada\\
Universidade de Brasília\\
Brasília DF, 70910-900, Brasil}
\and
\IEEEauthorblockN{Tiago Pereira Vidigal}
\IEEEauthorblockA{Mestrado em Computação Aplicada\\
Universidade de Brasília\\
Brasília DF, 70910-900, Brasil}
\and
\IEEEauthorblockN{William Oliveira Camelo}
\IEEEauthorblockA{Mestrado em Computação Aplicada\\
Universidade de Brasília\\
Brasília DF, 70910-900, Brasil}
}

% make the title area
\maketitle

\begin{abstract}
Problemas lineares modelam vários cenários de otimização. O uso de \textit{solvers} permite uma resolução rápida e padronizada por computadores. A função \textit{'scipy.optimize.linprog'} do pacote Python \textit{SciPy} foi usado em alguns exemplos para demonstrar seu uso. Sua flexibilidade e simplicidade permitiu facilmente modelar diversos problemas e sua eficiência se mostrou interessante com até 30 variáveis.
\end{abstract}

% keywords


\IEEEpeerreviewmaketitle
\section{Introdução} \label{sec_intro}

Problemas de otimização onde as equações relacionadas são lineares são chamados de problemas lineares. Diversas técnicas existem para sua resolução, como métodos gráficos, analíticos e tabulares \cite{livro}. Por serem problemas comuns, \textit{solvers} que automatizam o calculo são amplamente disponíveis.

O \textit{solver} \textit{'scipy.optimize.linprog'}, parte da biblioteca \textit{SciPy}, é uma alternativa Python para resolução automática de problemas lineares. A função foi implementada pela \textit{The SciPy community} e permite a modelagem de forma bem flexível. O método de resolução padrão é o \textit{interior-point} de Andersen, mas é possível escolher dentre as opções \textit{simplex} de Dantzig, \textit{revised simplex} de Bartels, \textit{HiGHS} de Huangfu et al (tendo as opções \textit{dual simplex} e \textit{interior-point}). \cite{solver}

O objetivo deste trabalho é apresentar alguns exemplos demonstrando o uso do \textit{solver 'scipy.optimize.linprog}. O único método de resolução utilizado é o \textit{interior-point} de Andersen.

\section{Desenvolvimento} \label{sec_des}

O uso do solver é avaliado em três exemplos. O objetivo é demonstrar a modelagem de problemas reais utilizando a formatação necessária para seu uso. O resultado de cada modelagem é apresentado na seção \ref{sec_result}.

A modelagem do problema para o \textit{solver} espera uma \textit{array} indicando os coeficientes das equações e limites envolvidos. O problema deve ser de minimização e as desigualdades devem estabelecer apenas limites superiores ao problema. Equações de igualdade também podem ser usadas caso se encaixem no problema.

\subsection{Exemplo 1: Carteira de Investimentos} \label{sec_ex1}

O primeiro exemplo é utilizando a questão 3.6-2 \cite{livro}. O modelo é:

\begin{equation}
\begin{aligned}
\min \quad & -Z = -50x_1 - 20x_2 - 25x_3\\
\textrm{unequals} \quad & 9x_1 + 3x_2 + 5x_3 \leq 500\\
  &5x_1 + 4x_2 + 0x_3 \leq 350\\
  &3x_1 + 0x_2 + 2x_3 \leq 150\\
\textrm{bounds} \quad & 0 \leq x_1 \\
  &0 \leq x_2 \\
  &0 \leq x_3 \leq 20 \\
\end{aligned}
\end{equation}

O problema, para ser usado no solver, é modelado e resolvido da seguinte forma:

\begin{lstlisting}[language=Python]
c = [-50, -20, -25]
A_ub = [[9, 3, 5],
        [5, 4, 0],
        [3, 0, 2]]
b_ub = [500, 350, 150]
x1_bounds = (0, None)
x2_bounds = (0, None)
x3_bounds = (0, 20)
bounds = [x1_bounds, x2_bounds, x3_bounds]

res = scipy.optimize.linprog(c=c,
        A_ub=A_ub, b_ub=b_ub,
        bounds=bounds)
\end{lstlisting}

\subsection{Exemplo 2: Escala de funcionários} \label{sec_ex2}

O segundo exemplo é utilizando a questão 3.6-3 \cite{livro}. O modelo é:

\begin{equation}
\begin{aligned}
\min \quad & Z = 100(6x_1 + 8x_2 + 7x_3 + 4x_4 + 9x_5 + 6x_6)\\
\textrm{equals} \quad & 1x_1 + 0x_2 + 0x_3 + 1x_4 + 0x_5 + 0x_6 = 300\\
  &0x_1 + 1x_2 + 0x_3 + 0x_4 + 1x_5 + 0x_6 = 200\\
  &0x_1 + 0x_2 + 1x_3 + 0x_4 + 0x_5 + 1x_6 = 400\\
  &1x_1 + 1x_2 + 1x_3 + 0x_4 + 0x_5 + 0x_6 = 400\\
  &0x_1 + 0x_2 + 0x_3 + 1x_4 + 1x_5 + 1x_6 = 500\\
\end{aligned}
\end{equation}

O problema, para ser usado no solver, é modelado e resolvido da seguinte forma:

\begin{lstlisting}[language=Python]
c = [600, 800, 700, 400, 900, 600]
A_eq = [[1, 0, 0, 1, 0, 0],
        [0, 1, 0, 0, 1, 0],
        [0, 0, 1, 0, 0, 1],
        [1, 1, 1, 0, 0, 0],
        [0, 0, 0, 1, 1, 1]]
b_eq = [300, 200, 400, 400, 500]

res = scipy.optimize.linprog(c=c,
        A_eq=A_eq, b_eq=b_eq)
\end{lstlisting}

\subsection{Exemplo 3: Problema de mistura} \label{sec_ex3}

O terceiro exemplo é utilizando a questão 3.6-4 \cite{livro}. O modelo é:

\begin{equation}
\begin{aligned}
\min \quad & -Z = 25(x_{11} + x_{12} + x_{13} + x_{14} + x_{15}) \\
  &\quad \quad +26(x_{21} + x_{22} + x_{23} + x_{24} + x_{25})\\
  &\quad \quad +24(x_{31} + x_{32} + x_{33} + x_{34} + x_{35})\\
  &\quad \quad +23(x_{41} + x_{42} + x_{43} + x_{44} + x_{45})\\
  &\quad \quad +28(x_{51} + x_{52} + x_{53} + x_{54} + x_{55})\\
  &\quad \quad +30(x_{61} + x_{62} + x_{63} + x_{64} + x_{65})\\
\textrm{unequals} \quad & -1(x_{11} +x_{12} +x_{13} +x_{14} +x_{15}) \leq -8\\
  &-1(x_{21} +x_{22} +x_{23} +x_{24} +x_{25}) \leq -8\\
  &-1(x_{31} +x_{32} +x_{33} +x_{34} +x_{35}) \leq -8\\
  &-1(x_{41} +x_{42} +x_{43} +x_{44} +x_{45}) \leq -8\\
  &-1(x_{51} +x_{52} +x_{53} +x_{54} +x_{55}) \leq -7\\
  &-1(x_{61} +x_{62} +x_{63} +x_{64} +x_{65}) \leq -7\\
\textrm{equals} \quad & 1(x_{11} +x_{21} +x_{31} +x_{41} +x_{51} +x_{61}) = 14\\
  &1(x_{12} +x_{22} +x_{32} +x_{42} +x_{52} +x_{62}) = 14\\
  &1(x_{13} +x_{23} +x_{33} +x_{43} +x_{53} +x_{63}) = 14\\
  &1(x_{14} +x_{24} +x_{34} +x_{44} +x_{54} +x_{64}) = 14\\
  &1(x_{15} +x_{25} +x_{35} +x_{45} +x_{55} +x_{65}) = 14\\
\textrm{bounds} \quad & 0 \leq x_{11} \leq 6,\quad x_{12} = 0,\quad\ \ \quad 0 \leq x_{13} \leq 6 \\
  &x_{14} = 0,\quad\ \ \quad 0 \leq x_{15} \leq 6,\quad x_{21} = 0 \\
  &0 \leq x_{22} \leq 6,\quad x_{23} = 0,\quad\ \ \quad 0 \leq x_{24} \leq 6 \\
  &x_{25} = 0,\quad\ \ \quad 0 \leq x_{31} \leq 4,\quad \leq x_{32} \leq 8 \\
  &0 \leq x_{33} \leq 4,\quad x_{34} = 0,\quad\ \ \quad 0 \leq x_{35} \leq 4 \\
  &0 \leq x_{41} \leq 5,\quad 0 \leq x_{42} \leq 5,\quad 0 \leq x_{43} \leq 5 \\
  &x_{44} = 0,\quad\ \ \quad 0 \leq x_{45} \leq 5,\quad 0 \leq x_{51} \leq 3 \\
  &x_{52} = 0,\quad\ \ \quad 0 \leq x_{53} \leq 3,\quad 0 \leq x_{54} \leq 8 \\
  &x_{55} = 0,\quad\ \ \quad x_{61} = 0,\quad\ \ \quad x_{62} = 0 \\
  &x_{63} = 0,\quad\ \ \quad 0 \leq x_{64} \leq 6,\quad 0 \leq x_{65} \leq 2 \\
\end{aligned}
\end{equation}

O problema, para ser usado no solver, é modelado e resolvido da seguinte forma:

\begin{lstlisting}[language=Python]
c = [25]*5+[26]*5+[24]*5+[23]*5+[28]*5+[30]*5
A_ub = [[-1]*5 +[0]*5 +[0]*5 +[0]*5 +[0]*5 +[0]*5,
         [0]*5+[-1]*5 +[0]*5 +[0]*5 +[0]*5 +[0]*5,
         [0]*5 +[0]*5+[-1]*5 +[0]*5 +[0]*5 +[0]*5,
         [0]*5 +[0]*5 +[0]*5+[-1]*5 +[0]*5 +[0]*5,
         [0]*5 +[0]*5 +[0]*5 +[0]*5+[-1]*5 +[0]*5,
         [0]*5 +[0]*5 +[0]*5 +[0]*5 +[0]*5+[-1]*5]
b_ub = [-8, -8, -8, -8, -7, -7]
A_eq = [[1, 0, 0, 0, 0]*6,
        [0, 1, 0, 0, 0]*6,
        [0, 0, 1, 0, 0]*6,
        [0, 0, 0, 1, 0]*6,
        [0, 0, 0, 0, 1]*6]
b_eq = [14, 14, 14, 14, 14]
x11_bounds = (0, 6); x12_bounds = (0, 0)
x13_bounds = (0, 6); x14_bounds = (0, 0)
x15_bounds = (0, 6); x21_bounds = (0, 0)
x22_bounds = (0, 6); x23_bounds = (0, 0)
x24_bounds = (0, 6); x25_bounds = (0, 0)
x31_bounds = (0, 4); x32_bounds = (0, 8)
x33_bounds = (0, 4); x34_bounds = (0, 0)
x35_bounds = (0, 4); x41_bounds = (0, 5)
x42_bounds = (0, 5); x43_bounds = (0, 5)
x44_bounds = (0, 0); x45_bounds = (0, 5)
x51_bounds = (0, 3); x52_bounds = (0, 0)
x53_bounds = (0, 3); x54_bounds = (0, 8)
x55_bounds = (0, 0); x61_bounds = (0, 0)
x62_bounds = (0, 0); x63_bounds = (0, 0)
x64_bounds = (0, 6); x65_bounds = (0, 2)
bounds = [x11_bounds, x12_bounds, x13_bounds,
          x14_bounds, x15_bounds, x21_bounds,
          x22_bounds, x23_bounds, x24_bounds,
          x25_bounds, x31_bounds, x32_bounds,
          x33_bounds, x34_bounds, x35_bounds,
          x41_bounds, x42_bounds, x43_bounds,
          x44_bounds, x45_bounds, x51_bounds,
          x52_bounds, x53_bounds, x54_bounds,
          x55_bounds, x61_bounds, x62_bounds,
          x63_bounds, x64_bounds, x65_bounds]

res = scipy.optimize.linprog(c=c,
       A_eq=A_eq, b_eq=b_eq, A_ub=A_ub, b_ub=b_ub,
       bounds=bounds)
\end{lstlisting}

\section{Resultados}\label{sec_result}

O resultados dos problemas modelados em \ref{sec_des} são apresentados a seguir. Um resumo do processamento, o valor de Z e os valores de X são listados. O exemplo 1 resultou em:
\begin{lstlisting}
Summary of results:
     con: array([], dtype=float64)
 message: 'Optimization terminated successfully.'
     nit: 5
   slack: array([1.01560659e-06, 7.16238674e-07,
                 3.14285717e+01])
  status: 0
 success: True

Z: 2904.761898856456
X: [26.19047614 54.76190465 19.99999996]
\end{lstlisting}

O exemplo 2 resultou em:
\begin{lstlisting}
Summary of results:
     con: array([9.24823041e-07, 6.14513795e-07,
                 1.23501030e-06, 1.23185009e-06,
                 1.54249705e-06])
 message: 'Optimization terminated successfully.'
     nit: 5
   slack: array([], dtype=float64)
  status: 0
 success: True

Z: 539999.9983378148
X: [1.39655395e-08 1.99999999e+02 1.99999999e+02
    2.99999999e+02 3.49164400e-09 1.99999999e+02]
\end{lstlisting}

O exemplo 3 resultou em:
\begin{lstlisting}
Summary of results:
     con: array([2.19345608e-09, 2.41093900e-09,
                 2.19378471e-09, 2.41274023e-09,
                 2.19379359e-09])
 message: 'Optimization terminated successfully.'
     nit: 6
   slack: array([9.99999998e-01, -1.34472167e-09, 
                 1.10000000e+01,  1.20000000e+01,
                -9.97330218e-10, -1.25582034e-09])
  status: 0
 success: True

Z: 1754.9999997146087
X:     
[[2.78890143 0. 2.78890142 0.         3.42219715]
 [0.         2. 0.         6.         0.        ]
 [4.         7. 4.         0.         4.        ]
 [5.         5. 5.         0.         5.        ]
 [2.21109857 0. 2.21109857 2.57780285 0.        ]
 [0.         0. 0.         5.42219715 1.57780285]]
\end{lstlisting}

O tempo de resolução de cada exemplo foi calculado pela média de 100 execuções. O valor encontrado, em segundos, foi:
\begin{itemize}
\item Exemplo 1: $0.0035194964 \pm 0.0019942994$ s
\item Exemplo 2: $0.0039004740 \pm 0.0022250296$ s
\item Exemplo 3: $0.0045470143 \pm 0.0022907777$ s
\end{itemize}

\section{Conclusão} \label{sec_concl}

O \textit{solver} apresentou os resultados dos problemas em tempo hábil utilizando um notebook pessoal, mesmo para o exemplo 3 com 30 variáveis. A facilidade de representar o problema com \textit{arrays} e a flexibilidade de juntar igualdades, desigualdades e limites demonstrou que é uma ótima alternativa para a resolução de problemas lineares.

% that's all folks
\bibliographystyle{IEEEtranN}
\bibliography{referencias.bib}
\end{document}
